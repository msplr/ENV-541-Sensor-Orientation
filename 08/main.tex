\documentclass{article}
\usepackage[utf8]{inputenc}
\usepackage{amsmath,amssymb}
\usepackage{graphicx}
\usepackage{float}
\usepackage{subcaption}
\usepackage{geometry}
\geometry{
    a4paper,
    total={170mm,257mm},
    left=20mm,
    right=20mm,
    top=20mm,
}
\usepackage{listings} % code listings
\lstset{framextopmargin=0pt,frame=lines}
\lstset{
    language=Matlab,
    basicstyle=\footnotesize\ttfamily,
    breaklines=true,
    tabsize=4,
    keepspaces=true,
    columns=flexible,
    % backgroundcolor=\color[gray]{0.9},
    frame=single,
    breaklines=true,%
    morekeywords={matlab2tikz},
    keywordstyle=\color{blue},%
    morekeywords=[2]{1}, keywordstyle=[2]{\color{black}},
    identifierstyle=\color{black},%
    stringstyle=\color{mylilas},
    commentstyle=\color{mygreen},%
    showstringspaces=false,%without this there will be a symbol in the places where there is a space
    numbers=left,
    numberstyle={\tiny \color{black}},% size of the numbers
    numbersep=9pt, % this defines how far the numbers are from the text
    emph=[1]{for,end,break},emphstyle=[1]\color{red}, %some words to emphasise
    %emph=[2]{word1,word2}, emphstyle=[2]{style},
}
\usepackage{color} %red, green, blue, yellow, cyan, magenta, black, white
\definecolor{mygreen}{RGB}{28,172,0} % color values Red, Green, Blue
\definecolor{mylilas}{RGB}{170,55,241}

\usepackage{siunitx}
\newcommand{\e}[1]{\times 10^{#1}} % nicer scientific notation

\title{ENV-541 Sensor Orientation\\Lab 8 - Kalman Filtering with simulated GPS data: constant acceleration (a = const.) model}
\author{Michael Spieler}
\date{November 23, 2018}

\begin{document}

\maketitle

\section*{Plots}

\section*{Questions}


\subsubsection*{I. Is the overall improvement of the \textit{positioning accuracy} via KF filtering better with
respect to the model of constant velocity (i.e. plot/compare $\sigma_{xy}^{KF_{emp.}}$ versus $\sigma_{xy}^{GPS_{emp.}}$)?
If yes/no – what do you think is the reason?}

\subsubsection*{II. Is the \textit{anticipated accuracy} ($\sigma_{xy}^{KF_{p}}$) different from that of the model of constant
velocity? Provide a justification.}

\subsubsection*{III. How does the \textit{distribution of velocity errors} ($v^{true} – v^{KF}$) compare to that obtained
with the constant velocity model? Can you identify and explain a pattern in these
differences for the model of constant acceleration?}

\subsubsection*{IV. Do the plotted histograms of innovations look like a normal distribution (visually)?
If yes/no, which case (from A or B in Task 4.4) is closer to normality and why?}

They look both normally distributed, with B a bit closer to normality (more symmetric).
The constant acceleration model fits well the test trajectory, since the acceleration in the map frame is only changing slowly.
Therefore the position prediction noise is directly coming from the GPS gaussian noise.

Case B is better because when the gain is stabilized the covariance is lower than than at start (high initial uncertainty).
This caused the filter to give less weight to the prediciton from the motion model.

\subsubsection*{V. Do you observe statistically significant changes for the position or velocity
empirical or anticipated accuracies when using \textit{faster prediction} (i.e. $dt_{KF}=0.1s$ vs.
$dt_{KF}=1s$)? \textit{Justify} your answer by reasoning.}

\subsubsection*{VI. Observe what happens when you first increase and then decrease the given value of
the system noise level ($\sigma_{\dot a}$) 10 times. How do you interpret the observed changes
in the filtered positions, velocities and the distribution of innovation, respectively?}

\newpage
\section*{Code}
\lstinputlisting{../lab8.m}

\end{document}
