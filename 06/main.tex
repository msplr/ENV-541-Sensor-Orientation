\documentclass{article}
\usepackage[utf8]{inputenc}
\usepackage{amsmath,amssymb}
\usepackage{graphicx}
\usepackage{float}
\usepackage{subcaption}
\usepackage{geometry}
\geometry{
    a4paper,
    total={170mm,257mm},
    left=20mm,
    right=20mm,
    top=20mm,
}
\usepackage{listings} % code listings
\lstset{framextopmargin=0pt,frame=lines}
\lstset{
    language=Matlab,
    basicstyle=\footnotesize\ttfamily,
    breaklines=true,
    tabsize=4,
    keepspaces=true,
    columns=flexible,
    % backgroundcolor=\color[gray]{0.9},
    frame=single,
    breaklines=true,%
    morekeywords={matlab2tikz},
    keywordstyle=\color{blue},%
    morekeywords=[2]{1}, keywordstyle=[2]{\color{black}},
    identifierstyle=\color{black},%
    stringstyle=\color{mylilas},
    commentstyle=\color{mygreen},%
    showstringspaces=false,%without this there will be a symbol in the places where there is a space
    numbers=left,
    numberstyle={\tiny \color{black}},% size of the numbers
    numbersep=9pt, % this defines how far the numbers are from the text
    emph=[1]{for,end,break},emphstyle=[1]\color{red}, %some words to emphasise
    %emph=[2]{word1,word2}, emphstyle=[2]{style},
}
\usepackage{color} %red, green, blue, yellow, cyan, magenta, black, white
\definecolor{mygreen}{RGB}{28,172,0} % color values Red, Green, Blue
\definecolor{mylilas}{RGB}{170,55,241}

\usepackage{siunitx}
\newcommand{\e}[1]{\times 10^{#1}} % nicer scientific notation

\title{ENV-541 Sensor Orientation\\Lab 6 - Coarse alignment of a high accuracy IMU}
\author{Michael Spieler}
\date{November 9, 2018}

\begin{document}

\maketitle

\section*{I. Read XSEA navigation-grade IMU data}

TODO: plot

\section*{II. Gyroscope: Earth rotation}


\subsubsection*{1. What are the differences you observe?}
\subsubsection*{2. How could these differences be used for estimating the approximate accuracy
of the gyroscopes? (If yes, specify the estimate.)}


\section*{III. Accelerometer: gravity vector}


\subsubsection*{4. What are the differences you observe?}
\subsubsection*{5. How could the observed differences be used for estimating the approximate
accuracy of your accelerometers? (If yes, specify the estimate).}


\section*{IV. Accelerometer leveling}


\subsubsection*{6. What is the value of roll in degrees and how does this correspond to the
reference value (real-time solution noted during acquisition*)?}
\subsubsection*{7. What is the value of pitch in degrees and how does this correspond to the
reference value (real-time solution noted during acquisition*)?}


\section*{V. Gyrocompassing}


\subsubsection*{8. What is the value of yaw in degrees and how does this correspond to the
reference value (real-time solution noted during acquisition*)?}
\subsubsection*{9. Do you recommend calculating the coarse alignment from instantaneous
observation (1 epoch) or do you suggest performing data averaging? Why?}
\subsubsection*{10. Is it possible to determine IMU’s latitude from the ‘leveled’ gyroscopic signal?
If yes, what is your estimate from these data, how it was obtained and how well
it corresponds to the reference value (real-time solution noted during
acquisition*)?}

\newpage
\section*{Code}
\lstinputlisting{../lab6.m}

\end{document}
