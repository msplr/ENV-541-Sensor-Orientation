\documentclass{article}
\usepackage[utf8]{inputenc}
\usepackage{amsmath,amssymb}
\usepackage{graphicx}
\usepackage{float}
\usepackage{subcaption}
\usepackage{geometry}
\geometry{
    a4paper,
    total={170mm,257mm},
    left=20mm,
    right=20mm,
    top=20mm,
}
\usepackage{listings} % code listings
\lstset{framextopmargin=0pt,frame=lines}
\lstset{
    language=Matlab,
    basicstyle=\footnotesize\ttfamily,
    breaklines=true,
    tabsize=4,
    keepspaces=true,
    columns=flexible,
    % backgroundcolor=\color[gray]{0.9},
    frame=single,
    breaklines=true,%
    morekeywords={matlab2tikz},
    keywordstyle=\color{blue},%
    morekeywords=[2]{1}, keywordstyle=[2]{\color{black}},
    identifierstyle=\color{black},%
    stringstyle=\color{mylilas},
    commentstyle=\color{mygreen},%
    showstringspaces=false,%without this there will be a symbol in the places where there is a space
    numbers=left,
    numberstyle={\tiny \color{black}},% size of the numbers
    numbersep=9pt, % this defines how far the numbers are from the text
    emph=[1]{for,end,break},emphstyle=[1]\color{red}, %some words to emphasise
    %emph=[2]{word1,word2}, emphstyle=[2]{style},
}
\usepackage{color} %red, green, blue, yellow, cyan, magenta, black, white
\definecolor{mygreen}{RGB}{28,172,0} % color values Red, Green, Blue
\definecolor{mylilas}{RGB}{170,55,241}

\usepackage{siunitx}
\newcommand{\e}[1]{\times 10^{#1}} % nicer scientific notation

\title{ENV-541 Sensor Orientation\\Lab 6 - Coarse alignment of a high accuracy IMU}
\author{Michael Spieler}
\date{November 9, 2018}

\begin{document}

\maketitle

\section*{I. Read XSEA navigation-grade IMU data}

The sensor data was loaded and the assigned sequence extracted.
Before the alignment the data is averaged over time and transformed from NWU to NED frame.
The noted reference values for verification were transformed from NWD to NED frame as well.

\section*{II. Gyroscope: Earth rotation}


\subsubsection*{1. What are the differences you observe?}

The observed Earth rotation rate is $7.2905091\e{-5} \si{rad/s}$ which has $-1.606\e{-8} \si{rad/s}$ difference to $\omega_e =  7.2921150\e{-5} \si{rad/s}$.

\subsubsection*{2. How could these differences be used for estimating the approximate accuracy
of the gyroscopes? (If yes, specify the estimate.)}

The overal accuracy of the individual gyroscopes can \textit{not} be determined from only the above difference.
This difference can come from multiple error sources which do not say anything about the accuracy of the individual 1 axis gyroscope.
The error can come from a constant bias, angular random walk, or misalignment of the axis.

Under the hypothesis that the sensor is properly calibrated (temperature compensation, misalignment correction, non-linearity, etc)

\section*{III. Accelerometer: gravity vector}
%accel: 9.8057706e+00 m/s^2, difference to g_EPFL 2.706e-04 m/s^2


\subsubsection*{4. What are the differences you observe?}

The measured gravity was $9.8057706 \si{m/s^2}$ with a difference of $2.706\e{-4} \si{m/s^2}$ to the given $g_{EPFL} = 9.8055 \si{m/s^2}$.

\subsubsection*{5. How could the observed differences be used for estimating the approximate
accuracy of your accelerometers? (If yes, specify the estimate).}

The overal accuracy of the individual gyroscopes can \textit{not} be determined from only the above difference.
With the same reasoning as for Question 2.

\section*{IV. Accelerometer leveling}
% roll -2.658 deg, pitch 2.465 deg
% reference: roll -2.653 deg, pitch 2.464 deg

\subsubsection*{6. What is the value of roll in degrees and how does this correspond to the reference value?}

The roll value is -2.658º, which is close to the reference value of -2.653º.

\subsubsection*{7. What is the value of pitch in degrees and how does this correspond to the reference value?}

The pitch value is 2.465º, which is close to the reference value of 2.464º.


\section*{V. Gyrocompassing}

\subsubsection*{8. What is the value of yaw in degrees and how does this correspond to the reference value?}

The yaw value is 42.256º. This is close to the reference value of 41.978º
% yaw 42.256 deg
% reference yaw 41.978 deg
% latitude 46.609 deg
% reference latitude 46.521 deg

\subsubsection*{9. Do you recommend calculating the coarse alignment from instantaneous
observation (1 epoch) or do you suggest performing data averaging? Why?}

No. Doing coarse alignment from one epoch is \textit{not} recommended because of the sensor noise.
Averaging over multiple samples can reduce the effect of white noise.

\subsubsection*{10. Is it possible to determine IMU’s latitude from the ‘leveled’ gyroscopic signal?
If yes, what is your estimate from these data, how it was obtained and how well
it corresponds to the reference value?}

Yes, the latitude can be estimated by calculating the angle of the angular rate 
vector to the xy plane of the local-level frame.
The estimated latitude is 46.609º, while the reference value is 46.521º.
\begin{equation*}
\phi = sin^{-1}\Big(\frac{-\omega^{l'}_{ie(3)}}{\omega_e}\Big)
\end{equation*}
with the leveled gyroscope signal $\omega^{l'}_{ie} = \mathbf{R}^{l'}_b \omega^b_{ie}$ and $\mathbf{R}^{l'}_b = \mathbf{R}_2(-pitch)\mathbf{R}_1(-roll)$

\newpage
\section*{Code}
\lstinputlisting{../lab6.m}

\subsection*{Output}
\begin{lstlisting}[language=]
>> lab6
1109_1522_PostProBinaryDecoded.imu contains 416036 epochs
Reading ... (data @ 200.00 Hz, sGr 0.000005 sAc 1.000000e-03) ... done in 0.4 sec.

Sensor norm:
gyro: 7.2905091e-05 rad/s, difference to w_ref -1.606e-08 rad/s
accel: 9.8057706e+00 m/s^2, difference to g_EPFL 2.706e-04 m/s^2

roll -2.658 deg, pitch 2.465 deg
reference: roll -2.653 deg, pitch 2.464 deg

yaw 42.256 deg
reference yaw 41.978 deg

latitude 46.609 deg
reference latitude 46.521 deg
\end{lstlisting}

\end{document}
